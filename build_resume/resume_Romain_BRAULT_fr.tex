%%%%%%%%%%%%%%%%%%%%%%%%%%%%%%%%%%%%%%%%%
% Freeman Curriculum Vitae
% XeLaTeX Template
% Version 2.0 (19/3/2018)
%
% This template originates from:
% http://www.LaTeXTemplates.com
%
% Authors:
% Vel (vel@LaTeXTemplates.com)
% Alessandro Plasmati
%
% License:
% CC BY-NC-SA 3.0 (http://creativecommons.org/licenses/by-nc-sa/3.0/)
%
%!TEX program = xelatex
% NOTICE: This template must be compiled with XeLaTeX, the line above should
% ensure this happens automatically but if it doesn't you will need to specify
% XeLaTeX as the engine in your editor or script
%
%%%%%%%%%%%%%%%%%%%%%%%%%%%%%%%%%%%%%%%%%
%
%------------------------------------------------------------------------------
%   PACKAGES AND OTHER DOCUMENT CONFIGURATIONS
%------------------------------------------------------------------------------
%
\documentclass[10pt]{article} % Font size, can be: 10pt, 11pt or 12pt
\usepackage[french]{babel}
\input{structure.tex} % Include the file that specifies the document structure
%
% Headers and footers can be added with the \lhead{} \rhead{} \lfoot{} \rfoot{}
% commands
% Example right footer:
\lfoot{\vspace{-.75cm}\color{headings}{\sffamily Dernière mise à jour: \today.}}
\rfoot{\vspace{-.75cm}\thepage~of~1}
%
%------------------------------------------------------------------------------
%
\begin{document}
\begin{paracol}{2} % Begin the multi-column environment
%
%------------------------------------------------------------------------------
%   NAME AND CURRICULUM VITAE TEXT
%------------------------------------------------------------------------------
%
\parbox[top][0.12\textheight][c]{\linewidth}{ %
% Parbox to hold the author name and CV text; fixed height to match the
% coloured box to the right, centred vertically and full line width
    \vspace{-2cm} % Reduce whitespace above the parbox to separate
                              % it from the main content
    \centering % Centre text
    {\sffamily\Huge Romain R. Brault}\\\medskip % Your name
    {\Huge\color{headings}\cvtextfont R\'esum\'e}
}
\switchcolumn % Switch to the next paracol column
%
%------------------------------------------------------------------------------
%   COLOURED CONTACT DETAILS BOX
%------------------------------------------------------------------------------
%
\parbox[top][0.12\textheight][c]{\linewidth}{ % Parbox to hold the colour box;
    % fixed height to match the name/CV text to the left, centred vertically
    % and full line width
    \vspace{-2cm} % Reduce whitespace above the parbox to separate
    % it from the main content
    \colorbox{shade}{ % Create the coloured box
        \begin{supertabular}{p{0.05\linewidth}|p{0.775\linewidth}} % Start a
        %table with two columns, the table will ensure everything is aligned
            \raisebox{-1pt}{\faHome} &
                Antony (92160),
                France\\ % Address
            %\raisebox{-1pt}{\faPhone} & +33 (0)6 82 14 72 05 (private) \\
            \raisebox{0pt}{\small\faEnvelope} &
                \href{mailto:mail@romainbrault.com}{ %
                      mail@romainbrault.com} \\ % Email address
            \raisebox{-1pt}{\faGithub} &
                \href{https://github.com/RomainBrault}{ %
                      RomainBrault} \\ % GitHub profile
            \raisebox{-1pt}{\faLinkedinSquare} &
                \href{https://www.linkedin.com/in/romain-brault}{ %
                      /in/romain-brault} \\ % LinkedIn profile
            %\raisebox{-1pt}{\faGraduationCap} &
                %\href{https://scholar.google.fr/citations?user=IFkrCzYAAAAJ}{ %
                %IFkrCzYAAAAJ} \\
        \end{supertabular}
    }
}
\end{paracol}
\vspace{-1.75cm}
\begin{center}
    \textbf{TL;\,DR:} Ingénieur de recherche dynamique et curieux, je suis à la
    recherche d'un poste d'expert en science des données et intelligence
    artificielle.
\end{center}
\vspace{-3.5mm}
\begin{paracol}{2}
%
%------------------------------------------------------------------------------
%   WORK EXPERIENCE
%------------------------------------------------------------------------------
%
\section{Postes}
%
% Blank \workposition command to add another job:
%
%\workposition{} % Duration
%{} % FT/PT (full time or part time)
%{} % Employer
%{} % Job title
%{} % Description
%
% All 5 parameters must be supplied but any can be empty if you don't need them
%
%------------------------------------------------------------------------------
%
\workposition{Actuel, depuis décembre 2018} % Duration
{TP} % FT/PT (full time or part time)
{ThereSIS, Thales SIX} % Employer
{Ing. de recherche} % Job title
{L'objectif de notre équipe est d'explorer, sélectionner et analyser des
  solutions à l'état de l'art en apprentissage artificiel (ML) et science des
  données pour les faire monter en maturité (TRL) pour répondre à
  des besoins clients. \par
  J'ai pu travailler sur diverses thématiques ML comme la prédiction de séries
  temporelles, la reconnaissance de la parole, du traitement de
  signal (audio et radars), etc. \par
  Mes travaux récent portent sur l'apprentissage automatique et profond de
  métriques pour l'audio. Je suis également impliqué dans des initiatives vistant
  à promouvoir l'utilisation de pratiques "MLOps" au sein de Thales.}\par %
\workposition{Actuel, depuis Décembre 2018} % Duration
{TP} % FT/PT (full time or part time)
{T\'el\'ecom Paris} % Employer
{Prof. Associé} % Job title
{Remplacements ponctuels pour des cours en ML (principalement sur les méthodes
à noyaux et l'apprentissage supervisé).}\par
\workposition{de octobre 2017 à octobre 2018} % Duration
{TP} % FT/PT (full time or part time)
{L2S, Centrale-Sup\'elec} % Employer
{Postdoc.} % Job title
{Mes recherches s'articulaient autour de l´apprentissage de fonctions d'ordre
  supérieur pour l'estimation de quantiles conditionels.
  Code 
  \href{https://bitbucket.org/RomainBrault/itl/}{TensorFlow} publié et travaux réalisés
  en colab avec le
  labo \href{https://portail.polytechnique.edu/cmap/fr}{CMAP} de l'École
  Polytechnique.  }\par % Description
% \workposition{September 2013, from Mai 2013}
% {FT}
% {Department of Computing, Imperial College London}
% {Master Thesis}
% {Worked on Finite Elements Method: modified
% \href{http://fluidityproject.github.io/publications/}{Fluidity}'s code to
% integrate bended elements to the \href{https://fenicsproject.org/}{FEniCS}
% project.}\par
%\workposition{July 2012, from June 2012}
%{FT}
%{CMLA, ENS Cachan}
%{Research Intern}
%{Implementation of a parallel lock-free SGD algorithm using OpenMP and SSE
%instructions.  Also worked on a CUDA GPGPU implementation.}\par
%\workposition{August 2011, from June 2011}
%{FT}
%{CEA${}^2$}
%{Research Intern}
%{Used Machine Learning techniques to calibrate the hyperparameters of a
%simulation code for nuclear reactions
%(\href{http://www.talys.eu/home/}{TALYS}).}\par
%
%------------------------------------------------------------------------------
%   TECHNICAL SKILLS
%------------------------------------------------------------------------------
%
\section{Compétences techniques}
%
% Example \tableentry{} command to add another line:
%
%\tableentry{Heading}{Content}{spaceafter}
%
% All 3 parameters must be supplied but any can be empty if you don't need them
% A "spaceafter" value in the third parameter will add some vertical space --
% this is to be used between headings
%
%------------------------------------------------
%
% ensure everything is aligned
%
%------------------------------------------------
%
\begin{itemize}[noitemsep]
    \item[ML]: \textbf{TensorFlow}, PyTorch, \textbf{Scikit-Learn},
      Pandas, Dash, \\ \textbf{kernel-methods}, Statistiques, Analyse fonctionelle,
    \item[HPC]: C/C++, CUDA, OpenMP, MPI, Culture Hardware,
    \item[Dev]: \textbf{Open-Source}, MLOps, Prefect, MLFlow, S3, \\ REST
      (FastAPI), DevOps, conteneurs (Docker/Podman), k8s,
    \item[*]: \textbf{\LaTeX}, VIM, Typage, Programation fonctionelle
    %\tableentry{Beginner}
  % \item \textbf{Python}, Poetry, Pandas, PyTorch, \textbf{Tensorflow},
    % Scikit-Learn, Multiprocessing, Prefect, MLFlow, Dash, ResAPI, ...
%%
%%------------------------------------------------
%%
    %\tableentry{Intermediate}
  % \item \texttt{C/C++}, Hardware knowledge, CUDA, OpenMP, MPI \textbf{CI/CD},
    % DevOps, \textbf{MLOps}, Containers, K8s, Open-source contribs, VIM
    % lover
%
%------------------------------------------------
%
    %\tableentry{Expert}
  % \item \LaTeX, Machine Learning, Statistics, Func. Analysis,
\end{itemize}
%
%------------------------------------------------
%
%
%
%------------------------------------------------------------------------------
%   SKILLS DESCRIPTION
%------------------------------------------------------------------------------
%
\section{Qualités personelles}
%
% Example \longformdescription{} command to add another section:
%
%\longformdescription{Heading}{Description}
%
%------------------------------------------------
%
\longformdescription{\textbf{Ouvert d'esprit}}{
Je m'attache à écouter les points de vue de tous sur un sujet pour prendre les
décisions les plus informés possible. Je tire beaucoup de satisfaction a
apprendre continuellement à travers discussions et enseignement.}\par
%
\longformdescription{\textbf{Passionné}}{
L'informatique et mathématiques m'intéressent depuis de longues
années. Les sciences des données ont la qualité d'être à
la croisée de nombreuses disciplines et encouragent l'apprentissage
permanent de nombreux sujets variés.}\par

\longformdescription{\textbf{Innovation "sans compromis"}}{
Dans me différent postes j'aprécie m'attaquer à des problèmes ardus avec
rigueur: étude de l'état de l'art, proposition de solutions innovantes pour
enfin étudier ces dernières empiriquement et théoriquement. Sans oublier de
fournir un code testé et documenté.}\par
%
%------------------------------------------------------------------------------
%
\switchcolumn % Switch to the next paracol column
%
%------------------------------------------------------------------------------
%   EDUCATION
%------------------------------------------------------------------------------
%
\section{Formation}
%
% Blank \educationentry{} command to add another degree:
%
%\educationentry{} % Duration
%{} % Degree
%{} % Honours, achievements or distinctions (e.g. first class honours)
%{} % Department
%{} % Institution
%
% All 5 parameters must be supplied but any can be empty if you don't need them
%
%------------------------------------------------------------------------------
%
\begin{supertabular}{rl} % Start a table with two columns, the table will
% ensure everything is aligned
%
%------------------------------------------------
%
    \educationentry{2013 -- 2017} % Duration
    {Docteur} % Degree
    {} % Honours, achievements or distinctions (e.g. first class honours)
    {Apprentissage Artificiel} % Department
    {\href{https://www.ibisc.univ-evry.fr/}{UEVE} \&
     \href{https://ltci.telecom-paristech.fr/}{T\'el\'ecom ParisTech}}
    % Institution
%
%------------------------------------------------
%
    \educationentry{2012-2013} % Duration
    {Master of Science} % Degree
    {} % Honours, achievements or distinctions (e.g. first class honours)
    {Computer Science} % Department
    {\href{http://www.imperial.ac.uk/computing}{Imperial College London}} %
    % Institution
%
%------------------------------------------------
%
    \educationentry{2010 -- 2013} % Duration
    {Diplôme d'ingénieur} % Degree
    {} % Honours, achievements or distinctions (e.g. first class honours)
    {Informatique} % Department
    {\href{http://www.ensiie.fr/}{ENSIIE, Paris-Saclay}}
    % Institution
%
%------------------------------------------------
%
    \educationentry{2010 -- 2011} % Duration
    {Licence} % Degree
    {} % Honours, achievements or distinctions (e.g. first class honours)
    {Mathématiques} % Department
    {\href{https://www.univ-evry.fr/accueil.html}{UEVE, Paris-Saclay}}
    % Institution
%
%------------------------------------------------
%
\end{supertabular}
%
%------------------------------------------------------------------------------
%   PUBLICATIONS
%------------------------------------------------------------------------------
%
\section{Publications}
%\printbibliography[heading=subbibliography,type=article,title={Publications}]
%\printbibliography[heading=subbibliography,type=inproceedings,title={Communication}]
%\printbibliography[heading=subbibliography,type=thesis,title={Thesis}]
%\printbibliography[keyword=review,title={To appear}]
4 publications dans des revus à comités de lecture (h-idx 3, 62 cit.) and 3 brevets
déposés à Thales. Voir Google scholar:
\href{https://scholar.google.fr/citations?user=IFkrCzYAAAAJ}{/citations?user=IFkrCzYAAAAJ}.
\medskip
%
%------------------------------------------------------------------------------
%   MAJOR RESEARCH PROJECT
%------------------------------------------------------------------------------
%
\section{Doctoral Research}
%
{\raggedright\textbf{``Régression à noyaux à valeurs opérateurs pour grands ensembles de données
"} \\}
Mes recherches doctorales furent motivés par mon désir de trouver des solutions
mathématiques à des problématiques réelles. J'aprécie particulièrement pouvoir
livrer un programe informatique testé et robuste avec des garantie théoriques.
Durant ma préparation de thèse j'ai eu l'occasion de donner des cours à
l'\href{https://www.univ-evry.fr/accueil.html}{UEVE} et
\href{https://www.telecom-paris.fr/}{T\'el\'ecom Paris}.
%
%------------------------------------------------------------------------------
%   AWARDS
%------------------------------------------------------------------------------
%
% \section{Distinctions}
%
% Example \tableentry{} command to add another line:
%
%\tableentry{Heading}{Content}{spaceafter}
%
% All 3 parameters must be supplied but any can be empty if you don't need them
% A "spaceafter" value in the third parameter will add some vertical space --
% this is to be used between headings
%
%------------------------------------------------
%
% \begin{supertabular}{rp{5.8cm}} % Start a table with two columns, the table
% will ensure everything is aligned
%
% ------------------------------------------------
%
    % \tableentry{2012}{\textbf{Best engineering project}}{}
    % \tableentry{}{\textit{Among the French Junior Entreprises}}{}
    % \tableentry{}{National reward to our team  for the
    % \href{https://www.altenrecrute.fr/blog-alten/evenements/alten-recompense-la-junior-entreprise-diese}%
    % {best engineering project} among the French Junior
    % Entreprises.}{spaceafter}
%
% ------------------------------------------------
%
    % \tableentry{2011}{\textbf{Top Achiever Award}}{}
    % \tableentry{}{\textit{Best Individual Project (out of 109)}}{}
    % \tableentry{}{Implemented a kernel SGD algorithm
    % (\href{http://users.cecs.anu.edu.au/~williams/papers/P172.pdf}{NORMA}) for
    % house market prediction in London.}{spaceafter}
%
% ------------------------------------------------

% \end{supertabular}
% \medskip
%
%------------------------------------------------------------------------------
%   LANGUAGES
%------------------------------------------------------------------------------
%
\section{Langues}
\textbf{Fran\c cais} (natif), \, \textbf{Anglais} (Courant), \\
\textbf{Portuguais} (intermédiaire), \, \textbf{Allemand} (notions)
%
%------------------------------------------------------------------------------
%   REFERENCES
%------------------------------------------------------------------------------
%
\section{References}
%
%\textit{References available on request}
%
%------------------------------------------------
%
% Example \tableentry{} command to add another line:
%
%\tableentry{Heading}{Content}{spaceafter}
%
% All 3 parameters must be supplied but any can be empty if you don't need them
% A "spaceafter" value in the third parameter will add some vertical space --
% this is to be used between headings
%
%------------------------------------------------
%
\begin{supertabular}{rl} % Start a table with two columns, the table will
% ensure everything is aligned
%
%------------------------------------------------
%
    \tableentry{}{\textbf{Pr.~Florence \textsc{d'Alch\'e-Buc}}}{}
    \tableentry{Poste}{Professeur}{}
    \tableentry{Employeur}{\href{https://ltci.telecom-paristech.fr/}%
                               {\textit{LTCI}},
                          \href{https://www.telecom-paristech.fr/} %
                               {\textit{T\'el\'ecom ParisTech}}}{}
    \tableentry{Mél}{\small\href{mailto:florence.dalche@telecom-paristech.fr}%
                      {florence.dalche@enst.fr}}{spaceafter}
%
%------------------------------------------------

\end{supertabular}
%
%\medskip % Extra whitespace before the next section
%
%------------------------------------------------------------------------------
%   HOBBYS
%------------------------------------------------------------------------------
%
\section{Hobbys}
Durant mon temps libre j'apprécie faire de l'escalade, des randonnées et du
vélo et aime également voyager. J'ai pratiqué le tir à l'arc et un peu de
photographie.
%------------------------------------------------------------------------------
%
\end{paracol}
%
%------------------------------------------------------------------------------
%
\vspace{-1.5mm}
\huge\centering\adforn{40}
%
\end{document}
