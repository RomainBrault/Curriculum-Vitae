\documentclass{tccv}
\usepackage[english]{babel}
\usepackage[a-1b]{pdfx}
\usepackage{url}
\usepackage{pythontex}
\usepackage{nth}

\begin{filecontents*}{\jobname.xmpdata}
    \Title{Curriculum Vitae}
    \Author{Romain R. Brault}
    \Keywords{CV \sep Machine Learning \sep Big Data \sep Deep Learning \sep
    Data Science}
\end{filecontents*}

\begin{document}

\part{\textsc{Romain Brault}}

\section{Professional Experience}

\begin{eventlist}

\item{Jan.~2014 --- Now}
     {\textbf{Operalib --- Paris-Saclay Center for Data Science}}
     {\large \textit{Machine learning, software engineering.}}

Main countributor and founder of
\href{https://github.com/operalib/operalib}{Operalib}. Operalib is a library
for structured learning and prediction for python. The idea is to predict
silmultaneously several targets while taking into account of the structures
between the targets.

\vspace{-5mm}\paragraph{}
Referee: \href{http://perso.telecom-paristech.fr/~fdalche/Site/index.html}{Pr.
Florence \textsc{d'Alch\'e-Buc}}.

\item{Oct.~2013 --- 2016}
     {\textbf{Teaching Assistant --- UEVE}}
     {}

\vspace{-5mm}
{Teaching Assistant: OCaml, Operating systems, Logic, C, Discrete
optimization.}

\item{Mai 2013 --- Sep.~2013}
     {\textbf{Reasearch intern at Imperial College London --- Faculty of
     Natural Sciences, Department of Mathematics.}}
     {\large \textit{Simulation, software engineering.}}

Worked on Firedrake: performance portable Automated code generation for
simulation sofware. Modified Fluidity to generate bended Lagrange elements for
the FEniCS Form Compiler.

\vspace{-5mm}\paragraph{}
Referee: \href{https://www3.imperial.ac.uk/people/david.ham/}{Dr. David
\textsc{Ham}}.


\item{Jun.~2012 --- Jul.~2012}
     {\textbf{Reasearch intern at Centre des Math\'ematiques et Leurs
     Applications --- CMLA, ENS Cachan.}}
     {\large \textit{Big Data and Machine Learning.}}


Implementation of a CPU parallel stochastic gradient descent using the OpenMP
library and the x86 SSE instruction set. Also worked on a CUDA accelerated
algorithm. Learned about big data challenges, high performance algorithm and
ranking methods.
\vspace{-5mm}\paragraph{}
Referee: \href{http://nvayatis.perso.math.cnrs.fr/}{Pr. Nicolas
\textsc{Vayatis}}.

% \item{Sep. 2011 -- Sep. 2012}
%      {\bf  Treasurer and Consultant at Junior-Enterprise Di\`ese.}
%      {\large \textit{Accounting, consulting and project management.} }

% Management of a work-group of 3 individuals among a team of 8 individuals.
% Accounted and budgeted the association during one year (about 70K€ revenue).
% Consultant in 3 smaller structures in France as an active auditor of the
% National Confederation of Junior Enterprises.

\item{Jun.~2011 --- Aug.~2011}
     {\textbf{Reasearch intern at French Alternative Energies and Atomic Energy
     Commission --- CEA}$\mathbf{{}^2}$.}
     {\textit{\large Contribution of Machine Learning for the calibration a
     nuclear reaction software.} }

Applied supervised learning technics (SVM) to calibrate a simulation code for
nuclear reactions (TALYS). Conception of statistical learning tools with R.
Used MPI to launch multiple instance of TALYS on TERA 100 with different inputs
parameters.

\vspace{-5mm}\paragraph{}
Referee: \href{mailto:pierre.dossantos-uzarralde@cea.fr}{Dr. Pierre
\textsc{Dossantos-Uzarralde}}.

\vspace{15cm}

\end{eventlist}

\personal%
    {15 rue Plumet, apt.~1 --- Paris 75015 (FR)}
    {+33 (0)6 82 14 72 05 }
    {romain.brault@telecom-paristech.fr}

\section{Education}

\begin{yearlist}
\item[Machine Learning]{2013 --- 2017}
     {Ph.~D.~. Thesis defense the \nth{3} Jul. 2017}
     {UEVE \& T\'el\'ecom-ParisTech}

\item[Computer Science]{2012 --- 2013}
     {Master of Science.}
     {Imperial College London.}

\item[Computer Science]{2010 --- 2013}
     {Dipl\^ome d'Ing\'enieur.}
     {ENSIIE, \'Evry, France.}

\item[Math]{2010 --- 2011}
     {Bachelor of Science.}
     {UEVE, \'Evry, France.}

\item[Math, Physics, Computer Science]{2008 --- 2010}
     {Preparatory classes to the Grandes \'Ecoles.}
     {MPSI/MP* Albert Schweitzer.}

\end{yearlist}


\section{Awards}

\begin{yearlist}

\item{2012}
     {Imperial College London: Machine Learning and Neural Computation.}
     {Best individual project out of 109.}

\item{2011}
     {Junior Enterprise Di\`ese: 2011 best engineering project.}
     {Rewarded by \textsc{Alten} and \textsc{Os\'eo} to our team of 10
     students.}

\end{yearlist}

\section{Communication Skills}

\begin{factlist}
\item{French:}{Mother tongue.}
\item{English:}{Fluent.}
\item{German:}{Basic knowledge.}
\end{factlist}

\section{Technical Skills}

\begin{factlist}

\item{Languages}
     {C, C++, cmake, Fortran, OCaml, Matlab, Maple, R, x86 assembly, Latex}

\end{factlist}

\cleardoublepage%
\section{Publications}
\begin{yearlist}
\item[R. Brault, M. Heinonen, F. d'Alch\'e-Buc]{2016}
     {Random Fourier Features for Operator-Valued Kernels}
     {Proceedings of 8th Asian Conference in Machine Learning}

\item[R. Brault, F. d'Alch\'e-Buc]{2017}
     {Operator Random Fourier Features: a generalization}
     {To be submited at JMLR}

\item[N. Goix, R. Brault, N. Drougard, M, Chiapino]{2017}
     {One Class Splitting Criteria for Random Forests with Application to
     Anomaly Detection}
     {Under review at ECML}

\end{yearlist}

\section{Communication acts}
\begin{yearlist}
\item[R. Brault, N. Lim, F. d'Alch\'e-Buc]{2016}
     {Scaling up Vector Autoregressive Models With Operator-Valued Random
     Fourier Features.}
     {Proceedings of 2nd ECML/PKDD Workshop on Advanced Analytics and Learning
     on Temporal Data.}

\item[R. Brault, F. d'Alch\'e-Buc]{2016}
     {Borne sur l'approximation de noyaux à valeurs opérateurs à l'aide de
     transformées de Fourier.}
     {Proceedings of 48th days of Statistics of the SFdS.}

\item[S. Varet, P. Dossantos-Uzarralde, N. Vayatis, R. Brault, and E.
Bauge.]{2011}
     {Experimental Covariances Contributions to Evaluated Cross Section
     Uncertainty Determination.}
     {Proceedings of the Second Workshop on Neutron Cross Section Covariances,
     Vienna.}
\end{yearlist}

\section{Personal interests}

\begin{itemize}
\item Archery, climbing, trekking,
\href{https://500px.com/romainbrault}{photography}.
\end{itemize}

\end{document}
